\documentclass[11pt]{article}
\usepackage{amsmath, amssymb, amscd, amsthm, amsfonts}
\usepackage{graphicx}
\usepackage{hyperref}
\usepackage{bm}

\oddsidemargin 0pt
\evensidemargin 0pt
\marginparwidth 40pt
\marginparsep 10pt
\topmargin -20pt
\headsep 10pt
\textheight 8.7in
\textwidth 6.65in
\linespread{1.2}

\title{Kepler's Laws and Orbital Mechanics}
\author{Nihar Annam \and Edward Chen \and Edison Huang \and Kesavan Venkatesh}
\date{December 2019}

\newtheorem{theorem}{Theorem}[section]
\newtheorem{lemma}[theorem]{Lemma}
\newtheorem{conjecture}[theorem]{Conjecture}
\newtheorem{corollary}[theorem]{Corollary}
% \newtheorem{theorem}{Problem}
\newcommand{\rr}{\mathbb{R}}

\theoremstyle{definition}
\newtheorem{defn}[theorem]{Definition}
\newtheorem*{prob}{Problem}

\newcommand{\al}{\alpha}
\DeclareMathOperator{\conv}{conv}
\DeclareMathOperator{\aff}{aff}

% \newcommand{\bf}{\mathbf}
\newcommand{\pder}[2][]{\frac{\partial#1}{\partial#2}}
\newcommand{\der}[2][]{\frac{\mathrm{d}#1}{\mathrm{d}#2}}
\begin{document}

\maketitle

% \begin{abstract}
% Purpose of this paper
% \end{abstract}

\section{Introduction}\label{section-introduction}
\subsection{History of Kepler's Laws}
Kepler had originally believed in Copernicus's model of the solar system. In 1601, after Tycho Brahe's death, Kepler inherited his data and upon analyzing his observations of Mars' orbit, discovered that planetary orbits followed the shape of an ellipse. This led to the development of his first law in 1605.

In 1602, while attempting to calculate the position of the Earth in its orbit with Brahe's data, found the second law - the radius vector of the orbital path of Earth describes equal areas in equal time.

Kepler published the first two laws in 1609 in his Astronomia Nova and added his third law in 1619 when he published his Harmonices Mundi.

In 1643, Godefroy Wendelin noted that Kepler's third law applied to the four brightest moons of Jupiter and in 1670, Nicolaus Mercator confirmed the second law in his Philosophical Transactions. Kepler's laws gained widespread acceptance towards the end of 17th century, when Newton published Principia and Gottfried Leibniz published his work on the laws. 

\subsection{Comparison with the Copernican Model of the Solar System}
Copernicus's Model stated that planets orbit around the sun in a circle at a constant speed. 
Kepler agreed with these statements if the eccentricities of the orbit is zero. However, the planetary orbits known at that time, notably Mars', had small eccentricities, not zero. Using the observations of the time, Kepler improved on Copernicus's model by stating that a planets orbit around a focal point in a ellipse at a constant area speed.

\subsection{The Statement}
Kepler's Laws are formally stated as follows: 
\begin{theorem}\label{KPL}
    Suppose a planet of mass $m$ is orbiting the sun, whose mass is $M$. Assume $M \gg m$.
    \begin{enumerate}
        \item The planet follows an elliptical trajectory around the sun. The sun is located at one of the foci of this curve. 
        \item The vector from the sun to the planet sweeps area at a constant rate with respect to time. 
        \item The square of the period of the planet's orbit is proportional to the cube of the orbit's semi-major axis. 
    \end{enumerate}
\end{theorem} 
% The paper is organized as follows. In Section 2, we will introduce some preliminary concepts that will be needed in the derivation of this theorem. Section 3 presents the derivation of Kepler's laws. From there, we use Section 4 to add some concrete examples of how Kepler's Laws are used in astrophysics. Section 5 is more supplementary, as it gives some extra theory for physics-oriented readers that expands on this survey. 
The usefulness of Kepler's Laws lies in its ability to quantitatively describe orbital motion. We can make theoretical predictions such as the famous \textit{Hohmann Transfer}– the optimal trajectory to transfer from the Earth to Mars orbit– consider the optimal position to provide an impulse to escape any orbit, or even estimate how long it would take the Earth to fall radially towards the Sun! On the other hand, in practice, Kepler's laws lay the foundation for astrophysics, enabling direct calculations of orbital velocities, periods, and more. Hopefully you're convinced that these laws matter; let's set out to prove them. \cite{Stewart}



\section{Preliminaries}\label{section-preliminaries}
\begin{defn}
The polar coordinate system defines any vector $\bm{x}$ in terms of $\bm{\hat{r}}$ and $\bm{\hat{\theta}}$. In Cartesian coordinates, these unit vectors are given as follows.
\begin{equation}
    \bm{\hat{r}} = \cos\theta \bm{\hat{x}} + \sin\theta \bm{\hat{y}}
\end{equation}
\begin{equation}
    \bm{\hat{\theta}} = -\sin\theta \bm{\hat{x}} + \cos\theta \bm{\hat{y}}
\end{equation}
A point in space is characterized as $\bm{x} = r\bm{\hat{r}}$ where $r \geq 0$ and $\theta$ is set to $0$ on the 'x-axis'.
\end{defn}
\begin{corollary}
\begin{equation}
    \der{t}\bm{\hat{r}} = \dot{\theta}\bm{\hat{\theta}}    
\end{equation}
\begin{equation}
    \der{t}\bm{\hat{\theta}} = -\dot{\theta}\bm{\hat{r}}
\end{equation}
In other words, the polar unit vectors are not necessarily fixed in time as the Cartesian ones are.
\end{corollary}
\begin{defn}
\label{conics}
A conic section is a curve $C$ such that the distance from a point $P$ on the curve to the focus $F$ is in proportion to the distance of $P$ to $l$, the directrix line. Mathematically, \begin{equation}
    \frac{|PF|}{|Pl|} = e
\end{equation}
where $e$ is some positive integer we call the \textit{eccentricity}. 
\end{defn}
\begin{theorem}
\label{polarconics}
In polar form, with one focus set at the origin, the equation of a conic section is \begin{equation}
    r = \frac{ed}{1 + e\cos\theta}
\end{equation}
where $e$ is the eccentricity and $x = d$ is the equation of the directrix. 
\end{theorem}
\begin{proof}
Setting the focus at the origin, we use \ref{conics} to say $r = e(d - r\cos\theta)$. Solving for $r$ yields the result. We can go a step further. Using \ref{polarconics}, the sum of the minimum and maximum values of $r$ are 
\begin{align}
    r_{min} + r_{max} = 2a &= ed\left(\frac{1}{1 + e} + \frac{1}{1 - e}\right)\\
    a &= \frac{ed}{1 - e^2}
\end{align}
and noting that $e = c / a = \frac{\sqrt{a^2 - b^2}}{a}$ 
\begin{align}\label{ed}
    ed = \frac{b^2}{a}.
\end{align}
\end{proof}
\begin{corollary}\label{ecc}
$e$ defines the particular type of conic section. 
\begin{enumerate}
    \item $e < 1 \implies $ ellipse. 
    \item $e = 1 \implies $ circle. 
    \item $e > 1 \implies $ hyperbola. 
\end{enumerate}
Note that $e > 0$ by definition. 
\end{corollary}
The importance of these results is that our derivation of Kepler's first law is aimed at producing a polar form of an ellipse from the equation of motion. One concern may be that this equation only works if the focus is set to be at the origin. However, when using this equation in practice, the placement of the origin is arbitrary; we typically care about quantities such as periods, velocities, etc. So we conveniently allow the focus where the sun (or a general central mass) to be seated at. 
\begin{theorem}\label{areaofellipse}
The area of an ellipse is \begin{equation}
    A_E = \pi ab
\end{equation}
where $a, b$ are the lengths of the semi-major and semi-minor axes, respectively. 
\end{theorem}
\begin{proof}
Parameterize the ellipse as \begin{align}
    \bm x = \left(a\cos t, b\sin t\right), 0 \leq t \leq 2\pi.
\end{align}
Green's theorem asserts the equality
\begin{align}
    \int_C \bm F \cdot \mathrm{d}\bm s = \int_S \left(\nabla \times \bm F\right) \cdot \mathrm{d}\bm S 
\end{align}
where $C$ is a closed curve that bounds $S$, both of which are taken in positive orientation. A special case of this equality is when the surface lies entirely in the $x-y$ plane. That is the case here: if $C$ is the boundary of an ellipse $S$ then 
\begin{align}
    \int_C \bm F \cdot \mathrm{d}\bm s = \int_S \left(\nabla \times \bm F\right) \cdot \bm{\hat{z}} \mathrm{d}A 
\end{align}
and if we define $\bm{F} = (-y, x)$ then 
\begin{align}
    \frac{1}{2} \int_C \left(x\der{t}{y} - y\der{t}{x}\right) \mathrm{d}t = \int_S \mathrm{d}A 
\end{align}
now substituting our parameterization in for $\bm F$
\begin{align}
    A_E &= \frac{1}{2} \int_0^{2\pi} \left(ab\cos^2t + ab\sin^2t\right) \mathrm{d}t \\
    &= \pi ab.
\end{align}
\end{proof}


\section{Derivation of Kepler's Laws}\label{section-derivation}
We will prove these laws out of order, doing them instead in a natural pedagogical order. 


\subsection{Kepler 2}\label{subsection-K2}
We have said that Kepler's Laws applies to planetary orbits around the sun. However, we will see that the second law in particular holds for \textit{any} orbital motion. Specifically, we will prove the stronger claim below. 
\begin{theorem}\label{K2}
The motion of a mass around a massive central body ($M \gg m$ assumption) has the condition that the radial vector $\bm x$ from the central mass to the orbiting body sweeps out equal areas in equal times. 
\end{theorem}
Before we tackle this, we need to introduce a few ideas.
\begin{defn}
\textit{Central-force motion} is the motion of an orbiting body under the sole influence of a massive central mass due to gravitational forces (e.g. a planet about the Sun, ignoring extraneous aspects such as friction or other planets). Quantitatively, a planet can be characterized to be undergoing this sort of motion if the net force on it is \begin{equation}
    \bm {F} = f(\bm{x})\bm{\hat{x}}
\end{equation}
where $f$ is a function of the position of the orbiting body. 
\end{defn}
\begin{defn}\label{angmom}
Let \begin{equation}
    \bm L := m \bm{x} \times \der{t}{\bm x}.
\end{equation} We will later call this the \textit{angular momentum} vector. 
\end{defn}
\begin{lemma}
$\bm L$ is conserved during central-force motion. 
\end{lemma}
\begin{proof}
If the derivative of $\bm L$ is equal to $\bm 0$, then $\bm L$ is conserved. Moving the constant $m$ ($m \neq 0$) over and taking the derivative (note that the product rule must be invoked)
\begin{align}
    \frac{1}{m}\der{t}{\bm L} &= \der{t}\left(\bm{x} \times \der{t}{\bm x}\right) \\
    &= \der{t}{\bm x}\times \der{t}{\bm x} + \bm{x} \times \der{t}\der{t}{\bm x}. \\
\end{align}
However by property of the cross product 
\begin{align} \label{selfcross}
    \bm{\hat{a}} \times \bm{\hat{a}} = \bm 0.
\end{align} So the first term goes to zero automatically. The second term is the reason this only applies to central-force motion: the acceleration vector (or the force vector, both point in the same direction by Newton's second law) is in the direction of $\bm{\hat{x}}$. Hence the second term also vanishes and we have achieved the desired relation 
\begin{equation}
    \der{t}{\bm L} = \bm 0.
\end{equation}
Technically, that completes the proof but let us take this a step further. If $\bm L$ is constant, we should be able to compute what it is. Building off \ref{angmom}
\begin{align} \label{expandL}
    \bm L &= m \left(r \bm{\hat{r}}\right) \times \der{t}\left(r\bm{\hat{r}}\right) \\
    &= m \left(r \bm{\hat{r}}\right) \times \left(\bm{\hat{r}} \der{t}{r} + r\bm{\hat{\theta}}\der{t}{\theta} \right) \\
    &= m r^2\bm{\hat{z}}\der{t}{\theta}
\end{align}
where the second step exploits properties of polar unit vectors and the last step invokes the distributive property of cross products. This is an important result to remember 
\begin{equation}
    \left|\bm{L}\right| = m r^2\der{t}{\theta}.
\end{equation}
\end{proof}
\begin{lemma}
Central-force motion occurs in a plane. 
\end{lemma}
\begin{proof}
Note the following two facts. 
\begin{align}
    \bm{x} \cdot \bm{L} &= 0\\
    \der{t}{\bm x} \cdot \bm{L} &= 0
\end{align}
They are both true because by definition $\bm{L}$ is perpendicular to both of those vectors (definition of cross-product). Another more algebraic way to see it is to make use of the identity 
\begin{align}\label{cycliccross}
    \bm a \cdot \left(\bm b \times \bm c\right) = \bm b \cdot \left(\bm c \times \bm a\right) = \bm c \cdot \left(\bm a \times \bm b\right)
\end{align}
as well as \ref{selfcross}. These two facts combined show that the position and velocity are perpendicular to the same vector which is constant throughout the motion. It necessarily follows that the motion lies in a plane with normal vector $\bm L$.
\end{proof}
Now we are ready to prove \ref{K2}. 
\begin{proof}
Define a function $A(\Delta \theta)$ that computes the area swept out by $\bm x$ as a function of the polar angle in the plane. Using a double integral, we see 
\begin{align}
    A(\Delta \theta) &= \int_{\theta_0}^{\theta}\int_{0}^{r} r'\mathrm{d}r'\mathrm{d}\theta' \\
    A(\Delta \theta) &= \int_{\theta_0}^{\theta} \frac{1}{2}r^2 \mathrm{d}\theta' \\
    \implies \der{\theta}{A} &= \frac{1}{2}r^2 \\
\end{align}
and so by chain rule
\begin{align}
    \der{t}{A} &= \left(\der{t}{\theta}\right)\left(\der{\theta}{A}\right) \\
    &= \frac{1}{2}r^2\der{t}{\theta}. \\
\end{align}
But that term is really familiar \dots it's the magnitude of the angular momentum (up to a constant factor)! Explicitly 
\begin{equation}\label{rateofarea}
    \der{t}{A} = \frac{1}{2m}\left|\bm{L}\right|.
\end{equation}
That actually finishes this proof. Since $\left|\bm L\right|$ is constant during central-force motion, the rate at which area is swept out is as well. Notice that we made no reference to any particular qualities of the orbiting system; this rule holds for all central-force motion!
\end{proof}


\subsection{Kepler 1}\label{subsection-K1}
There are actually two methods to prove this result. The derivation here will use multi-variable calculus methods; the second method is presented in \ref{section-orbmech}.
\begin{theorem}
The motion of a planet of mass $m$ around a massive central body like the Sun of mass $M$ ($M \gg m$) will follow an elliptical trajectory.
\end{theorem}
All the preliminary ideas were introduced in the previous subsection. The only new thing is just a small new notation: 
\begin{align}
    \bm v = \der{t}{\bm x} \\
    \bm a = \der{t}\der{t}{\bm x}.
\end{align}
Assuming you have read the previous work, let's dive right in. 
\begin{proof}
Let's define 
\begin{align}
    \bm \ell = \frac{1}{m}\bm{L}
\end{align}
just for convenience in this proof. Look back to our expansion of $\bm L$ in \ref{expandL}. If we didn't immediately simplify the expressions, we would have 
% \begin{align}
%     \bm \ell &= r\bm{\hat{r}} \times \left(\bm{\hat{r}}\der{t}{r} + r \der{t}{\bm{\hat{r}}}\right) \\
%     &= r^2\left(\bm{\hat{r}} \times \der{t}{\bm{\hat{r}}}\right).
% \end{align}
\begin{align}
    \bm \ell = r^2\left(\bm{\hat{r}} \times \der{t}{\bm{\hat{r}}}\right).
\end{align}
Now we make a slight leap: what if we consider $\bm{a} \times \bm{\ell}$? Newton's law says 
\begin{equation}
    \bm a = -\frac{GM}{r^2} \bm{\hat{r}}
\end{equation}
so
\begin{align}
    \bm{a} \times \bm{\ell}  &= -\frac{GM}{r^2} \bm{\hat{r}} \times r^2\left(\bm{\hat{r}} \times \der{t}{\bm{\hat{r}}}\right) \\
    &= -GM \bm{\hat{r}} \times \left(\bm{\hat{r}} \times \der{t}{\bm{\hat{r}}}\right) \\
    &= -GM\left(\left(\bm{\hat{r}} \cdot \der{t}{\bm{\hat{r}}}\right)\bm{\hat{r}} - \left(\bm{\hat{r}} \cdot \bm{\hat{r}}\right)\der{t}{\bm{\hat{r}}}\right) \\
    &= GM\der{t}{\bm{\hat{r}}}.
\end{align}
The third step uses the vector triple product identity. 
But notice that 
\begin{align}
    \der{t}{\left(\bm v \times \bm \ell\right)} = \bm{a} \times \bm{\ell}
\end{align}
which implies 
\begin{align}
    \bm{v} \times \bm{\ell} = GM\bm{\hat{r}} + \bm c
\end{align}
for some constant vector $\bm c$. Now we use a sort of trick: take the dot product of both sides with $\bm x$.
\begin{align}
    \bm{x} \cdot \left(\bm{v} \times \bm{\ell}\right) &= Gm \bm{x}\cdot \bm{\hat{r}} + \bm{x}\cdot\bm{c} \\ 
    &= GMr + rc\cos\theta \\
\end{align}
Rearranging the equality 
\begin{align}
    r = \frac{\bm{x} \cdot \left(\bm{v} \times \bm{\ell}\right)}{GM + c\cos\theta} \\
\end{align}
and invoking \ref{cycliccross} we see 
\begin{align}\label{resultK1}
    r = \frac{\ell^2}{GM + c\cos\theta}.
\end{align}
We are essentially done at this point but to completely match \ref{polarconics} let $e := \frac{c}{GM}$ and $d := \frac{\ell^2}{c}$. \\
Notice that this argument so far is completely general: it holds for any orbital motion (the reason being we have just used mathematics so far). We can narrow it to planetary motion with \ref{ecc}. Since planets follow closed paths (the Earth doesn't just fly out its orbit), we know $e < 1$ so the orbit is an ellipse.
\end{proof}
\subsection{Kepler 3}\label{subsection-K3}
\begin{theorem}
The square of the period of the motion of a mass around a massive central body ($M \gg m$ assumption) is proportional to the cube of the length of the semi-major axis. 
\end{theorem}
\begin{proof}
Remember the result stated in \ref{rateofarea}. Rewriting the right-hand side and integrating both sides
\begin{equation}
    \int_{0}^{A_E} \mathrm{d}A = \int_{0}^{T}\frac{\ell}{2}\mathrm{d}t
\end{equation}
where $A_E$ and $T$ are the ellipse's area and period, respectively. We use \ref{areaofellipse} to simplify
\begin{align}
    A_E = \pi ab = T\frac{\ell}{2} \\
    T = \frac{2\pi ab}{\ell}.
\end{align}
From \ref{resultK1}, 
\begin{equation}
    ed = \frac{\ell^2}{GM}
\end{equation}
so substituting this and \ref{ed} into our expression for the period
\begin{equation}
    T^2 = \frac{4\pi^2}{GM}a^3.
\end{equation}

\end{proof}

\section{Orbital Mechanics}\label{section-orbmech}
\subsection{Effective Potential and Energy Conservation}
Here's a quick note on how physicists approach the orbital motion problem. Instead of manipulating forces and vectors, they invoke \textit{conservation of energy}. 
\begin{theorem}
By conservation of energy, the energy $E$ is conserved in the orbit. It is given as follows. 
\begin{equation}
    E = \frac{m}{2}\left(\bm v \cdot \bm v\right) + V(r)
\end{equation}
where $V(r) = -\frac{GMm}{r}$ is the potential function for gravitational attraction.
\end{theorem}
We aren't going to prove that this quantity is conserved but intuitively it should make sense: we only have conservative forces so the sum of the kinetic and potential energy should be conserved as well. What is interesting is what happens if you expand the dot product. We omit the details of this expansion and cite the final result (it can be done using the definitions given in the derivation above). 
\begin{corollary}\label{conservationofenergy}
\begin{equation}
    E = \frac{m}{2}\left(\der{t}{r}\right)^2 + \frac{L^2}{2m r^2} + V(r)
\end{equation}
\end{corollary}
\begin{defn}
We define the effective potential as \begin{equation}
    V_e = \frac{L^2}{2m r^2} + V(r).
\end{equation}
\end{defn}
There are a couple uses of this definition. Nominally, it absorbs the angular momentum term so that the new energy expression looks like a simple sum of kinetic and potential energies. In addition, plotting the effective potential can give a good idea how the orbiting body moves around the central mass: explicitly, it will achieve equilibrium at the critical points of $V_e$ and exist in the region allowed by the energy $E$. For example, minimizing $V_e$ for a planetary orbit gives the radius of circular motion, the most stable orbital motion. 

\subsection{Binet's Equation}
We will take an alternate approach to deriving Kepler's first law, in the process reaching Binet's equation \cite{Tong}. Because of \ref{conservationofenergy}, we can take a time derivative of $E$ and set it to $0$. Let's denote $\dot r$ and $\ddot r$ to be the first and second derivatives of the radial distance with respect to time (similar for other variables). Doing so and simplifying gives
\begin{align}
    0 = m\ddot r - \frac{L^2}{mr^3} + \der{r}{V} \\
\end{align}
but $F(r) = -\der{r}{V}$ by definition (in physics) so 
\begin{align}
    F(r) = m\ddot r - \frac{L^2}{mr^3}.
\end{align}
Now comes the standard yet unmotivated trick to this approach: introduce 
\begin{align}
    u = \frac{1}{r}.
\end{align} It comes out of nowhere but it really makes the math tons easier.
\begin{align}
    \dot r = \dot \theta \der{\theta}{r} = \frac{\ell}{r^2}\der{\theta}{r} = -\ell \der{\theta}{u} \\
    \ddot r = -\ell^2 u^2 \der{\theta}{}\der{\theta}{u} \\
\end{align}
and so miraculously we can simplify the orbit equation to the following. 
\begin{theorem}
Binet's equation for a general force law is 
\begin{equation}
    \der{\theta}{}\der{\theta}{u} + u = -\frac{1}{m\ell^2u^2}F(1/u).
\end{equation}
\end{theorem}
This is easily solvable for inverse-square laws and it is clear that the polar equation of a ellipse pops up again. So, granted with an unmotivated substitution, the physics approach to Kepler's first law is much faster and cleaner. 
% \subsection{Two Body Problem}
% \subsection{Bertrand's Theorem}
\subsection{An Earth-Sun Collision}
We pose this as an intriguing problem.
\prob Suppose the Earth suddenly stops orbiting the Sun and radially falls toward it. Roughly how long would it take before the collision to occur? We assume for simplicity the initial orbit was circular. \cite{Morin}\\
\\
A trick we can use to understand physical situations is push them to the extreme cases. For this problem, a line of reasoning might go as follows. Start with a circular orbit. Now slowly squeeze the orbit, all the while keeping it an ellipse (so Kepler's first law is still satisfied), to the point where the orbit is essentially a straight line back and forth. Consider this the orbit the Earth follows as it falls toward the Sun. We can calculate the period of this orbit: here $a = R_0/2$ where $R_0$ is the initial orbital radius. That is enough to calculate the period because Kepler's Third Law asserts 
\begin{equation}
    T^2 = \frac{4\pi^2}{GM}R_0^3.
\end{equation}
Note however that we only want $T/2$! We only care about the part where the Earth falls \textit{toward} the Sun. Hence
\begin{equation}
    t = \left(\frac{\pi^2}{8GM}R_0^3\right)^{1/2}.
\end{equation}
Astronomical data suggests this is $\sim 60$ days.

% \subsection{Hohmann Transfer}

\section{Further Reading}
Some topics we left out due to both accessibility and time-constraints are: removing the $M \gg m$ assumption (what physicists call the \textit{two-body problem}), Bertrand's Theorem, and the mathematics of the Hohmann Transfer.


\bibliographystyle{alpha}
\bibliography{references} % see references.bib for bibliography management

\end{document}


%http://web.mit.edu/12.004/TheLastHandout/PastHandouts/Chap03.Orbital.Dynamics.pdf